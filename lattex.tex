

\noindent 

\noindent 

\noindent 

\noindent 

\noindent 



\noindent \textbf{                                                                                                   }

\noindent                                                                                                             

\noindent Literature and the group discussion, and begin generating hypotheses. When writing the data analysis, each topic identified should be categorized under similar themes. When describing significant themes that emerged, samples from the online transcripts should be provided as supporting evidence. The article by Rezabek (2000) in Forum for Qualitative Research (at http://qualitative-research.net/fqs/fqs-eng.htm) provides an example of how to write the data analysis for a Net-based focus group. 

\noindent       A thorough review of the literature should also provide the e-researcher with a solid basis of information to compare with the themes and topics that have emerged as well as to clarify issues that are both congruent and incongruent with the literature. The literature can either be integrated with the themes and topics or written as a separate section, usually titled "discussion".

\noindent       Finally, when reporting focus group results, in addition to an analysis of themes and topics, a description of the selection producers should be provided. This should include the criteria for sampling: how the participants were identified; the percentage that agreed to participate, as well as the percentage that eventually did participate; and demographic data (i.e., age, gender, education, socioeconomic status, etc.).

\noindent      Once the data have been analyzed, the e-researcher should have a greater understanding of the topic under investigation and be able to define the research question(s) for further study---the most common aim of focus groups in the field of education.

\noindent \textbf{Online Resources}

\begin{enumerate}
\item  Carter MeNamera maintains an excellent list of readings, resources, and tips relating to focus group process at http://www.mapnp.org/library/grp\_skill/focusgrp/foucusgrp.htm.

\item  The commercial firm Net.surveys provides an online focus group hosting service that can recruit participants and provides virtual rooms where online participants can be exposed to a variety of multimedia Web-based resources for comment or discussion. See http://www.web-surveys.net/net.probe/.

\item  Bob Dick provides a nice summary and tips for conducting structured focus groups online at http://www.scu.edu.au/schools/gcm/ar/arp/focus.html.
\end{enumerate}

\noindent 

\noindent \textbf{TIPS FOR FACILITATING A SUCCESSFUL NET-BASED FOCUS GROUP}

\begin{enumerate}
\item \textbf{ }Observe or participate in a Net-based focus group.

\item  Invite participants who are comfortable with communicating on the Internet.

\item  Anticipate an attrition rare of about 50 percent.

\item  Choose and word your questions carefully.

\item  Plan your probing questions ahead of time, but be prepared to change them based on the group's discussion.

\item  Write a friendly welcome letter and set a climate where participants feel free to contribute frank and candid remarks.

\item  Use nonevaluative feedback through nondirective probes.

\item  Follow up with a phone call to silent participants.
\end{enumerate}

\noindent 

\noindent 

\noindent                                                                                                            FOCUS GROUPS      119

\noindent 

\begin{enumerate}
\item  End the group discussion with a post-session debriefing.

\item  Thank all participants.
\end{enumerate}

\noindent 

\noindent 

\noindent \textbf{SUMMARY}

\noindent 

\noindent       Focus groups are generally used for in-depth and collaborative exploration of important issues and to redefine the research topic and the research questions. While Net-based focus groups have met with mixed results, their advantages over face-to-face focus groups are clear: they reduce or eliminate participation and cost barriers as well as overcome certain power issues between participants. However, effective Net-based focus groups will require a skilled morderator who can probe without breaking the flow or the pace of the group in an online enviroment.

\noindent          \textbf{\_\_\_\_\_\_\_\_\_\_\_\_\_\_\_\_\_\_\_\_\_\_\_\_\_\_\_\_\_\_\_\_\_\_\_\_\_\_\_\_\_\_\_\_\_\_\_\_\_\_\_\_\_\_\_\_\_\_\_\_\_\_\_\_\_\_\_             }

\noindent REFERENCES

\noindent Fowler, F.J.(1995). Improving survey questions. Design and evaluation. London: sage. Glesne, C., \& Peskin, A. (1992). Becoming qualitative researchers: An introduction. White Plains, NY: longman.

\noindent Jacoboson, P. (1997). On-line focus groups: Four approaches that work. Qurik's Marketing Research Review, Article Number 0245.

\noindent Mertens, D.M.(1998). Research methods in education and psychology: Integrating diversity with quantitative and qualitative approaches. Thousand Oaks, C.A: Sage.

\noindent Neuman, W.L.(2000). Social research methods. Qualitative and quatative approaches (4th ed.). Toronto, ON: Allyn \& Bacon.

\noindent Patton, M. Q.(1987). How to use qualitative methods in evaluation. London: Sage.

\noindent Patton, M. Q.(1990). Qualitative evaluative and research methods (2nd ed.). London: Sage.

\noindent Rezabek, R.J.(2000). Online focus groups: Electronic discussions for research. Forum for Qualitative Research. 1\eqref{GrindEQ__1_}. [Online]. Available: http://qualitative-research.net/fqs/fqs-eng.htm.

\noindent Silverman, G. (2000). How to get beneath the surface in focus groups. Market Navigation, Inc: [Online]. Available: http://www.mnav.com/bensurf.htm.

\noindent Silverman, G., \& Zukergood, E. (2000). Everything in moderation. Market Navigation, Inc. [Online]. Available: http://www.mnav.com/evmod.htm.

\noindent Stewart, D. W., \& Shamdasani, P.N. (1998). Focus group research: Exploration and discovery. In L. Bickman and D.J Rog(Eds.), Handbook of applied social research methods(pp. 505-526). London: Sage.

\noindent Van Nuys, D.(1999). Online focus groups save time, money. Silicon Valley / San Jose Business Journal, November. [Online]. Available: http://sanjose.bcentral.com/sanjose/stories/1999/11/29/smallb4.html.

\noindent 

\noindent \textbf{                                                                                            CHAPTER NINE}

\noindent                                    \_\_\_\_\_\_\_\_\_\_\_\_\_\_\_\_\_\_\_\_\_\_\_\_\_\_\_\_\_\_\_\_\_\_\_\_\_\_\_\_\_\_\_\_\_\_\_\_\_\_\_\_\_\_\_\_\_\_\_\_\_\_\_\_\_\_\_\_\_

\noindent             

\noindent                                                                               \textbf{NET-BASED CONSENSUS TECHNIQUES}

\noindent 

\noindent 

\noindent 

\noindent 

\noindent Bis reptita placent

\noindent The thing that please are those tht are asked for again and again.

\noindent Horace (65-8 BC)

\noindent 

\noindent 

\noindent 

\noindent 

\noindent 

\noindent Creating, sustaining, and improving social enviroments are challenging tasks that involve aesthetic, cultural, and affective considerations, as well as rational and scientific- based decision making. Much of the information needed to make informed policy decisions resides in the minds of professionals, administrators, researchers, taxpayers, and students. However, the views of these exports often are not readily available to the e-researcher and/or sometimes they provide conflicting advice. There are techniques to solicit and sample this individual knowledge (notably interviewers and surveys), however, sometimes a consensus is needed to arrive at a single best solution. The use of the Delphi Method, Nominal Group Technique, and Consensus Development Conference, as well as other experimental techniques have proved to be effective means to achieve this kind of goal. The purpose of consensus techniques is to systematically solicit export opinion to facilitate the clarification of issues---even in the absence of a group consensus(Linstone \& Turoff, 1975). This method  typically uses a carefully designed program of successive individual questioning generally conducted with written questionnaires.

\noindent       The primary reason for using consensus techniques is to structure a group communication process that creates useful results based on a consensus by exports in the field. Specifically, it is not the nature of this method that determines its appropriateness in a research study; rather, it is the circumstances of the research that neccessitates this kind of group communication process. Linstone and Turoff(1975) identified specific circumstances in which a researcher,  wishing to obtain group consensus, may find other kinds of communication processes (such as a focus group or interviews) too restrictive. Some of these circumstances include:

\noindent 

\noindent 
\[120\] 

